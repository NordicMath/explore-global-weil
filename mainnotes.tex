\documentclass[a4paper]{amsart}


%%% --- From the original amsart template --- %%%

\newtheorem{theorem}{Theorem}[section]
\newtheorem{lemma}[theorem]{Lemma}

\theoremstyle{definition}
\newtheorem{definition}[theorem]{Definition}
\newtheorem{example}[theorem]{Example}
\newtheorem{xca}[theorem]{Exercise}

\theoremstyle{remark}
\newtheorem{remark}[theorem]{Remark}

\numberwithin{equation}{section}



%%% --- New environments added by us --- %%%

\theoremstyle{plain}
\newtheorem{conjecture}{Conjecture}[section]
\newtheorem{exercise}{Exercise}[section]
\newtheorem{problem}{Problem}[section]
\newtheorem{application}{Application}[section]
\newtheorem{construction}{Construction}[section]
\newtheorem{proposition}[theorem]{Proposition}
\newtheorem*{corollary}{Corollary}
\newtheorem{propdef}[theorem]{Proposition-Definition}

\theoremstyle{remark}
\newtheorem*{note}{Note}



%%% --- Packages (all added by us) --- %%%

% TODO: Do we need all these? Remove one-by-one and compile.
% Note that some of these may not be compatible with amsart; see the Author Handbook link above.

\usepackage[english]{babel}
\usepackage[utf8]{inputenc}

\usepackage{amssymb}
\usepackage{graphicx}

\usepackage[colorinlistoftodos]{todonotes}
\usepackage{hyperref}
\usepackage{tikz-cd}
\usepackage{relsize}
\usepackage[makeroom]{cancel}

\usepackage{xifthen}

% Packages for tikz
\usepackage{tikz,ulem}
\usepackage{adjustbox}
\usetikzlibrary{arrows}

%\usepackage{showkeys}

%%% --- End of packages --- %%%



%%% --- New commands added by us --- %%%

\newcommand{\N}{\mathbb{N}}
\newcommand{\Z}{\mathbb{Z}}
\newcommand{\bbP}{\mathbb{P}}
\newcommand{\PP}{\mathbb{PP}}
\newcommand{\Q}{\mathbb{Q}}
\newcommand{\R}{\mathbb{R}}
\newcommand{\C}{\mathbb{C}}
\newcommand{\Fp}{\mathbb{F}_p}
\newcommand{\Fq}{\mathbb{F}_q}

\newcommand{\defhl}[1]{\textbf{#1}}

\newcommand{\twopartdef}[4]
{
	\left\{
		\begin{array}{ll}
			#1 & \mbox{if } #2 \\
			#3 & \mbox{} #4
		\end{array}
	\right.
}

\newcommand{\threepartdef}[6]
{
	\left\{
		\begin{array}{lll}
			#1 & \mbox{if } #2 \\
			#3 & \mbox{if } #4 \\
			#5 & \mbox{} #6
		\end{array}
	\right.
}

\newcommand{\Mod}[1]{\ (\text{mod}\ #1)}



\begin{document}

\title{Working notes on $m$-functions}
\maketitle

\tableofcontents

\section{Intro and summary}

We aim to collect all results obtained on $m$-functions and the various "helper" functions, including $h$, $c$, $\Psi$ and $g$.

An important part of this project includes the code base and the many different ways of plotting $m$ and the other functions. The plots and animations are collected in a Google Drive folder, while much of the code currently (Nov 2018) exists on local machines.



\subsection{Ideas and work in progress}

Right now (Nov 4th, 2018) it seems like the snake plot might be the most promising plot method, but we need to refine and test all the plotting technology further, while also doing more work on the theory side, especially on the $g$ function and the global functional equation (GFE).

\begin{enumerate}
\item IF we could understand (e.g. via numerical experiments) how the $h$-function looks for a given $m$-function, then we could maybe use this knowledge to DEFINE $m$ outside the unit circle. This would probably imply the pairing conjecture, and could possibly lead to a direct proof of the global functional equation for the underlying $L$-function, without relying on modular or automorphic forms. The reason for thinking this is that for Dirichlet characters, the pairing conjecture is essentially equivalent to the fact that $\chi$ is a character, which is in turn essentially equivalent to the GFE for the Dirichlet $L$-function.
\item In all the problems that follow it might be the case that Fourier analysis could help us. A conjecture is that for any $m$-function, the Fourier coefficients of $h(1, \theta)$ are rational numbers (perhaps with a simple formula).
\item Plan: Try to first understand $h(1, \theta)$ via the study of $c(r, \theta)$ as $r$ approaches $1$. Under some reasonable assumptions on $m$ it seems like $c(r, \theta)$ approaches $h(1, \theta)$, using that $m(1/z) \approx m(\bar{z})$ near any point on the unit circle where $m$ is continuous. Note that $c$ is defined on the open unit disc (unconditionally) but $c$ is not an analytic function.
\item Work first with Dirichlet characters, since in this case we know what the $m$-function is in the entire complex plane. This should also clarify which of our definitions and conjectures require the $m$-function to have real coefficients.
\item A good idea (?): Try to plot $\Psi(\theta)$ for different $m$-functions, maybe via the $c$-function as $r$ approaches 1.
\item Vague idea: Can we use power series inversion (applied to the denominator) to understand $c$ better?
\end{enumerate}


\section{Definitions}

\begin{definition}
  The $m$-function of $z$ associated to an arithmetic $f : \N \to \C$ is defined by the inifite sequence:
  $$m(z; f) = \sum_{n = 1}^{\infty} f(n) z^{n - 1}$$
\end{definition}

\begin{definition}
For any $m$-function $m(z)$,we define the associated $h$-function as
$$ h(z) = \frac{m(\frac{1}{z})}{m(z) }  $$
(which of course makes sense only if the $m$-function is defined outside of the unit circle).
\end{definition}

\begin{definition}
For any $m$-function $m(z)$,we define the associated $c$-function as
$$c(z) = \frac{m(\bar{z})}{m(z)}$$

\end{definition}

We write $h(r, \theta)$ for the $h$-function as a function of $z$ in polar coordinates, and similarly for $c$.

\begin{lemma}
(Do we need to assume real coefficients here??) We have
$$ \vert h(1, \theta) \vert = 1 $$
for all $\theta$.
\end{lemma}

\begin{definition}
Motivated by the previous lemma, we define
$$  \Psi(\theta) = \arg h(1, \theta)   $$
(we need to decide which branch of $\arg$ we use).
\end{definition}


\section{Background on complex analysis}

A hope is that there is a theorem which immediately implies that the number of poles on the unit circle of any $m$-function is finite, and that the number of zeroes on the unit disc is also finite. I (Andreas) am currently reading the book of Gamelin to find out, but we might need to ask on MathOverflow or search elsewhere.

An analytic (or meromorphic) function should be completely determined by its values on the real line, or on the unit circle (or any other curve segment). This means that if we understand for example $h$ on the unit circle, we can in principle reconstruct $h$ on the entire complex plane. BUT I (Andreas) don't know exactly how to do this. One possibility could be to use the Cauchy-Riemann equations, which we record here in their polar coordinate version.

\begin{proposition}
Assume that $f(z) = u(z) + i v(z) $ is an analytic function with real part $u$ and imaginary part $v$. Then we have
$$  \frac{\partial u}{\partial r} = \frac{1}{r} \frac{\partial v}{\partial \theta}$$
and
$$ \frac{1}{r} \frac{\partial u}{\partial \theta} = - \frac{\partial v}{\partial r} $$
\end{proposition}

\section{Plotting m-functions}


\subsection{Different methods for plotting}

There are many ways of plotting the $m$-function and its auxiliary functions.

\begin{enumerate}
\item Snake
\item \ldots (discuss and complete)
\end{enumerate}

Here are some further ideas:

\begin{enumerate}
\item IF we could find the zeroes of $m$ on the unit circle first, and write $p(z)$ for the polynomial (normalized via $p(0) = 1$) with those zeroes, then we could apply all of the previous plotting methods to $p(z)/m(z)$ instead. Now the former poles will become zeroes, and there are no obnoxious poles at all in the plots. An example is the elliptic curve 11a, where $p(z)$ MIGHT POSSIBLY be equal to $1-z^2$.
\item If we compose with arctan in front of any absolute value plot, we avoid the problem that values near poles go to infinity. Instead, the graph will look approximately linear (if the pole is a simple pole) and approximately quadratic (for a double pole) etc. See the arctan folder on Google Drive, and in particular IMG 0218 compared with IMG 0219. Note also that the coefficient in the numerator will affect the slope in a way that looks inversely proportional. Maybe we can modify the arctan function slightly to make the visual effects easier to interpret. We may also use a tangent to the graph near the unit circle, using the fact that when passing a simple pole, the graph is convex so that a tangent will always cross the vertical line $r=1$ at a point UNDER $\frac{\pi}{2}$, while when passing a higher-order pole, the concavity means that the same tangent will cross the vertical line at a point ABOVE $\frac{\pi}{2}$.
\end{enumerate}



\subsection{Convergence and bounds}

\begin{definition}
  We say that $m$ has proper convergence when the sum defining $m(z; f)$ absolutely converges whenever $|z| < 1$. All $m$-functions here have proper convergence unless otherwise stated.
\end{definition}

\begin{proposition}
  An $m$-function has proper convergence if and only if there exists an $N$ and a monomial $Bn^d$ such that $|f(n)| \le Bn^d$ for all $n \ge N$.
\end{proposition}

\begin{proof}
  The sum is absolutely convergent if and only if all tails of the function absolutely converge. Given an $N$ and a polynomial $P(x)$, look at the following tail:
  $$\sum_{n = N}^{\infty} |f(n)z^{n - 1}| = \sum_{n = N}^{\infty} |f(n)| |z|^{n - 1} \le \sum_{n = N}^{\infty} Bn^d |z|^{n - 1}$$
  We apply the Cauchy ratio test, and see that:
  $$L = \lim_{n \to \infty} \left|\frac{B(n + 1)^d |z|^{n}}{Bn^d |z|^{n - 1}}\right| = \lim_{n \to \infty} |z|\left|\frac{(n + 1)^d}{n^d}\right| = \lim_{n \to \infty} |z|\left|1 + \frac{ 1}{n}\right|^d = |z|$$
  From this, we see that whenever $|z| < 1$ we have absolute convergence of the $m$-function, which is what we wanted to show. For the other direction... \todo{how?}

\end{proof}

\section{Results on the $h$-function and the $c$-function}

Let's try to work out what the CR equations say (all of this paragraph might be completely wrong for some stupid reason). I think it is true that for the $h$-function, if we write $h = u+ iv$ and restrict attention to what happens on the unit circle, we get
$$ u(\theta) = \cos \Psi(\theta) \quad \quad \textrm{and} \quad \quad v(\theta) =  \sin \Psi(\theta) $$
from which we can determine $\frac{\partial u}{\partial \theta}$ and $\frac{\partial v}{\partial \theta}$. This should (using the chain rule) tell us what happens very near the unit circle as we vary $r$. But I don't see how to get to any differential equations that determine $h$ on the entire complex plane, BUT then again I haven't thought this through thoroughly at all.

\section{Plotting $h$ and $c$}




\section{Hadamard factorization and the $g$ function}

Add the relevant theorem here.

\section{The example of the Riemann $m$-function}

From the Riemann zeta function we get the $m$-function

$$ m(z) = \frac{1}{1-z}   $$

\begin{lemma}
For this $m$-function, we have
$$h(z) = -z$$
and
$$\Psi(\theta ) = \theta + \pi \pmod {2\pi}$$

\end{lemma}
Note that this gives a zig-zag function with nice Fourier coefficients.

\begin{proof}
The first statement follows immediately by a short computation. For the second statement, we have $h(1, \theta) = -e^{i \theta} = e^{i (\theta + \pi)} $
\end{proof}

\section{General results for Dirichlet characters}

It will be super-interesting to see what happens with $h$ and $c$ and $\Psi$ for Dirichlet characters.

\begin{proposition}
For any non-trivial Dirichet character $\chi$, the $h$-function is given by $h(z) = -\chi(-1) z^2$.
\end{proposition}
\begin{proof}
Since $\chi$ is periodic (with period $N$, say), we can write
$$ m(z) = \frac{1 + \chi(2) z + \ldots + \chi(N-1) z^{N-2} + \chi(N) z^{N-1}}{1-z^N} $$
and since $\chi(N) = \chi(0) = 0$ while $\chi(N-1)= \chi(-1) = \pm 1 \neq 0$, the degree of the numerator is $N-2$.

Expanding by $\chi(-1) z^N $ in the second step, and then using repeatedly that $\chi$ is multiplicative and periodic, we get
\begin{align*}
    m(\frac{1}{z}) &= \frac{1+\chi(2)/z + \chi(3) / z^2 + \ldots + \chi(N-2) / z^{N-3} + \chi(N-1) /z^{N-2} }{1-(1/z)^N}  \\
        &= \frac{z^2 \cdot \big( \chi(-1)z^{N-2} + \chi(-2) z^{N-3} + \ldots \chi(2-N) z + \chi(1-N) \big)}{\chi(-1) (z^N - 1 )} \\
        &= \frac{z^2}{- \chi(-1)} \cdot \frac{ \chi(N-1) z^{N-2} + \chi(N-2) z^{N-3} + \ldots + \chi(2) z + 1}{1-z^N} \\
        &= \frac{z^2}{-\chi(-1)} \cdot m(z) \\
        &= -\chi(-1) \cdot z^2 \cdot m(z)
\end{align*}
and the result follows.
\end{proof}

Remark: I think the proof also essentially proves the pairing conjecture, but we have to check carefully and write it down. The point is that the reverse polynomial of the numerator of $m$ is obtained from the original numerator just by substituting $1/z$ in place of $z$, and then multiplying by $\chi(-1) z^{N-2}$. But we have to check what happens with cancellation. Note that the Riemann example has just a single pole, so it cannot satisfy the pairing conjecture.

\begin{lemma}
We have $\prod_{j=1}^{N-1} (1-\omega^j) = N$.
\end{lemma}

\begin{proposition}
The residue of $m(z)$ at $\omega$ is given by
$$ Res(m, \omega) = - \frac{\tau(\chi)}{N}   $$
where $\tau(\chi)$ is the Gauss sum of $\chi$.
\end{proposition}

\begin{proof}
Define $f(z) = \chi(1) z + \chi(2) z^2 + \ldots + \chi(N-1) z^{N-1}$. Then $f(\omega)$ is the Gauss sum, and
$$m(z) = \frac{f(z)}{z-z^{N+1}}$$
which gives the result after a small computation.
\end{proof}
We could verify the lemma, then verify the proposition numerically, and also investigate what happens at other poles.

Together with \url{https://en.wikipedia.org/wiki/Gauss_sum} this will help us understand which roots of unity are poles and which ones are not.

\section{Examples of Dirichlet characters}

Unless otherwise stated, we consider the "Generator 0" given by Sage.

\subsection{Mod 3}

We get $m(z) = \frac{1}{1+z+z^2}$. The real plot on $[-1, 1]$ has minimum $\frac{1}{3}$ for $z=1$, and maximum $\frac{4}{3}$, for $z=-\frac{1}{2}$.

\subsection{Mod 4}



\subsection{Mod 5}


\subsection{Mod 15}

\section{The structure of m-functions}
We need a clearer purpose for this section...
\begin{theorem}
We have the following explicit formulas for values of the $m$-function of a Dirichlet product:
  $$m(z; A \oplus B) = \sum_{n = 1}^\infty z^{n - 1} A_n m(z^n; B) = \sum_{k = 1}^\infty z^{n - 1} B_n m(z^n; A)$$
\end{theorem}


\begin{proof}
  First we show that our m-function has proper convergence. Note that:
  $$(A \oplus B)_n = \sum_{ab = n} A_a B_b$$
  This gives us, via the triangle equality and bounds for $|A_a| \le c a^d$ and $|B_b| \le c' b^{d'}$ that:
  $$|(A \oplus B)_n| = \left|\sum_{ab = n} A_a B_b\right| \le \sum_{ab = n} |A_a| |B_b| \le \sum_{ab = n} c a^d c' b^{d'} $$
  $$ = cc' \sum_{ab = n} a^d b^{d'} \le cc' \sum_{ab = n} (ab)^{\max(d, d')} = cc' \sum_{ab = n} n^{\max(d, d')} $$
  $$ = cc' \tau(n) n^{\max(d, d')} \le cc' n n^{\max(d, d')} = cc' n^{1 + \max(d, d')}$$
  Note that this could be improved a lot, for instance by picking a more intelligent bound for $\tau(n)$ than $n$, but this is sufficient for us. We now know that $m((A \oplus B)_n)$ has proper convergence, and we can mess about with the order by which we sum. The definition of $m$-functions gives:
  $$m(z; A \oplus B) = \sum_{n = 1}^\infty z^{n - 1} \sum_{ab = n} A_a B_b = \sum_{n = 1}^\infty \sum_{ab = n} A_a B_b z^{n - 1}$$
  Since we have absolute convergence, we permute the order in the following manner:
  $$ = \sum_{\substack{n = 1\\ ab = n}}^\infty A_a B_b z^{n - 1} = \sum_{a, b \ge 1} A_a B_b z^{ab - 1} = \sum_{a \ge 1} \sum_{b \ge 1} A_a B_b z^{ab - 1}$$
  Now note that:
  $$z^{ab - 1} = z^{ab - a + a - 1} = z^{a(b - 1) + a - 1} = (z^a)^{b - 1} z^{a - 1}$$
  We insert this and get the following elegant rewriting:
  $$ = \sum_{a \ge 1} \sum_{b \ge 1} A_a B_b (z^a)^{b - 1} z^{a - 1} = \sum_{a \ge 1} A_a z^{a - 1} \sum_{b \ge 1} B_b (z^a)^{b - 1} = \sum_{n \ge 1} A_n z^{n - 1} m(z^n; B)$$
  This is the desired result. For the other one, swap $A$ and $B$ in the entire argument. 
  
\end{proof}

\begin{note}
  This does not give explicitly the new coefficients of $A \oplus B$, but it for a fixed $z$ we get that: 
  $$m(z; A \oplus B) = m(z; A \boxtimes m(z^n; B))$$
  
\end{note}

\section{General results for Quadratic number fields}

\begin{theorem}
  For any quadratic number field, we have $h(z) = \pm z^2$ where the exact sign is given in the proof.
\end{theorem}

\begin{proof}
  Note first that there is a Dirichlet character such that $F = 1 \oplus \chi$, where $1$ is the Riemann zeta function and $\chi$ a Dirichlet character. By theorem $\ref{???}$, we see that:
  $$m(z; F) = m(z; 1 \oplus \chi) = \sum_{n \ge 1} (1)_n z^{n - 1} m(z^n; \chi) = \sum_{n \ge 1} z^{n - 1} m(z^n; \chi)$$
  Inserting $1/z$, we get:
  $$m(1/z; F) = \sum_{n \ge 1} (1/z)^{n - 1} m((1/z)^n; \chi) = \sum_{n \ge 1} z^{1 - n} m(1/(z^n); \chi) $$
  Now, notice that, from before, $m(1/z; \chi) = -\chi(-1)z^2m(z; \chi)$, which gives:
  $$ = \sum_{n \ge 1} z^{1 - n} -\chi(-1)(z^n)^2m(z^n; \chi) = -\chi(-1)\sum_{n \ge 1} z^{1 - n} z^{2n} m(z^n; \chi)$$
  $$ = -\chi(-1)\sum_{n \ge 1} z^{1 - n + 2n} m(z^n; \chi) = -\chi(-1)\sum_{n \ge 1} z^{n + 1} m(z^n; \chi)$$
  $$ = -\chi(-1) z^2 \sum_{n \ge 1}  z^{n - 1} m(z^n; \chi) = -\chi(-1) z^2 m(z;1 \oplus \chi) = -\chi(-1) z^2 m(z;F)$$
  This is the desired result. 
\end{proof}

\section{Mellin transform}
In this section, we prove the BSD conjecture. Let $a_n$ be some numbers and:
$$f(x) := \sum_{n = 1}^\infty a_n e^{-nx}$$
We take the Mellin-transform: 
$$\mathcal{M}\{f(x)\}(s) = \int_0^\infty x^{s - 1} f(x) dx = \int_0^\infty x^{s - 1} \sum_{n = 1}^\infty a_n e^{-nx} dx$$
$$ = \sum_{n = 1}^\infty a_n \int_0^\infty x^{s - 1} e^{-nx} dx$$
Let $u = nx$. Then $dx = \frac{1}{n} du$.
$$ = \sum_{n = 1}^\infty a_n \int_0^\infty \left(\frac{u}{n}\right)^{s - 1} e^{-u} \frac{1}{n} du$$
$$ = \sum_{n = 1}^\infty a_n \frac{1}{n} \left(\frac{1}{n}\right)^{s - 1} \int_0^\infty u^{s - 1} e^{-u} du$$
$$ = \sum_{n = 1}^\infty \frac{a_n}{n^s} \Gamma(s) = L(s) \Gamma(s)$$
The Mellin invserse gives us: \todo{verify hypotheses}
$$f(x) = \frac{1}{2\pi i}\int_{c - i \infty}^{c + i \infty} x^{-s} \Gamma(s) L(s)ds$$
However, note also that:
$$e^{-x}m(e^{-x}) = f(x)$$
Now, let $a_n$ be coeffs of some elliptic curve of conductor $N$. Then we have the functional equation:
$$\Lambda(s) := N^{s/2} \Gamma_\C(s) L(s) = 2 \left(\frac{2\pi}{\sqrt{N}}\right)^{-s} \Gamma(s) L(s) = w\Lambda(2 - s)$$
Let:
$$E(N) = \frac{2\pi}{\sqrt{N}}$$
Note now that:
$$ f(E(N)) = f\left(\frac{2\pi}{\sqrt{N}}\right) = e^{-E(N)} m\left(e^{-E(N)}\right) = \frac{1}{2\pi i} \int_{c - i \infty}^{c + i \infty} \left(\frac{2\pi}{\sqrt{N}}\right)^{-s} \Gamma(s) L(s)ds$$
$$ = \frac{1}{2\pi i} \int_{c - i \infty}^{c + i \infty} \Lambda(s) ds = \frac{w}{2\pi i}\int_{c - i \infty}^{c + i \infty} \Lambda(2 - s) ds$$
Let $c = 1$, \todo{verify this is legal} and we get:
$$ = \frac{1}{2\pi i} \int_{-\infty}^{\infty} \Lambda(1 + ui) \frac{1}{i}du = \frac{w}{2\pi i}\int_{-\infty}^{\infty} \Lambda(1 - ui) \frac{1}{i}du$$
$$ = \frac{-1}{2\pi} \int_{-\infty}^{\infty} \Lambda(1 + ui) du = \frac{-w}{2\pi}\int_{-\infty}^{\infty} \Lambda(1 - ui) du$$
$$ = \frac{w}{2\pi}\int_{-\infty}^{\infty} \Lambda(1 + ui) du$$
This means we get:
$$-1 \int_{-\infty}^{\infty} \Lambda(1 + ui) du = w\int_{-\infty}^{\infty} \Lambda(1 + ui) du$$
When $w = 1$, this implies:
$$\int_{-\infty}^{\infty} \Lambda(1 + ui) du = 0$$
Which again means: 
$$f(E(N)) = e^{-\frac{2\pi}{\sqrt{N}}} m\left(e^{-\frac{2\pi}{\sqrt{N}}}\right) = 0$$


\section{Ramblings}

EVERYTHING HERE IS WRONG! BECAUSE $-s$ not $s$ and $(xM(x))' = m(x)$ not $xm'(x) = M(x)$


\subsection{Rambling1}

We have the following main functions using a multiplicative function $a_k$:

$$m(t) = \sum_{n = 1} a_nt^{n - 1} = 1 + a_2t + a_3t^2 + \ldots$$

$$Z_t(t, p) = \sum_{n = 0} a_{p^n} t^n = 1 + a_pt + a_{p^2}t^2 + \ldots$$

$$L(s) = \sum_{n = 1} a_n n^{-s} = 1 + \frac{a_2}{2^s} + \frac{a_3}{3^s} + \ldots$$

$$Z_s(s, p) = \sum_{n = 0} a_{p^n} p^{-sn} = 1 + \frac{a_p}{p^n} + \frac{a_{p^2}}{p^{2s}} + \ldots$$

Known relations:

$$Z_s(s, p) = Z_t(p^{-s}, p)$$
Euler product:
$$L(s) = \prod_p Z_s(s, p)$$
Mellin transform:
$$L(s)\Gamma(s) = \int_0^{\infty} x^{s - 1} e^{-x} m(e^{-x})dx$$

Deductions:

THIS IS WRONG!!!!!!!!!!!!!!!!!!!!!!!!!!!!!!!!!!!!!!!!!!!!!!!!!!!!!!!!!!!!!!!!!!!!!!!!!!!!!!!!!!!!!!!!!!!!!!!!!!!!!!!!!!!!!!!!!!!!!!!!!!!!!!!!!!!!!!!!!!!!!!!!!
ITS $x^{-s}$


Mellin inverse gives:
$$e^{-x}m(e^{-x}) = \frac{1}{2\pi i}\int_{c - i\infty}^{c + i\infty}x^{-s} \Gamma(s)L(s)ds$$
Setting $y = e^{-x}$ and $x = -\log(y)$
$$y m(y) = \frac{1}{2\pi i}\int_{c - i\infty}^{c + i\infty}(-\log(y))^{-s} \Gamma(s)L(s)ds$$
And then:
$$m(x) = \frac{1}{2\pi i x} \int_{c - i\infty}^{c + i\infty}(-\log(x))^{-s} \Gamma(s)L(s)ds$$
$$m(x) = \frac{1}{2\pi i x} \int_{c - i\infty}^{c + i\infty}e^{i\pi s}\log(x)^{-s} \Gamma(s)L(s)ds$$
Setting in:
$$m(x) = \frac{1}{2\pi i x} \int_{c - i\infty}^{c + i\infty}e^{i\pi s}\log(x)^s \Gamma(s)\prod_p Z_s(s, p)ds$$
$$m(x) = \frac{1}{2\pi i x} \int_{c - i\infty}^{c + i\infty}e^{i\pi s}\log(x)^s \Gamma(s)\prod_p Z_t(p^{-s}, p)ds$$
I think this is an analogue of the Euler product!

REVIEW THIS IS WRONG!!!!! 

Back to this:
$$m(x) = \frac{1}{2\pi i x} \int_{c - i\infty}^{c + i\infty}e^{i\pi s}\log(x)^s \Gamma(s)L(s)ds$$
We use this to find $M(t) = t m'(t)$. We get:
$$m'(x) = \left(\frac{1}{2\pi i x}\right)' \int_{c - i\infty}^{c + i\infty}e^{i\pi s}\log(x)^s \Gamma(s)L(s)ds + \frac{1}{2\pi i x} \left(\int_{c - i\infty}^{c + i\infty}e^{i\pi s}\log(x)^s \Gamma(s)L(s)ds\right)'$$
$$m'(x) = -\frac{1}{2\pi i x^2} \int_{c - i\infty}^{c + i\infty}e^{i\pi s}\log(x)^s \Gamma(s)L(s)ds + \frac{1}{2\pi i x} \int_{c - i\infty}^{c + i\infty}e^{i\pi s}\left(\log(x)^s\right)' \Gamma(s)L(s)ds$$
$$m'(x) = -\frac{1}{2\pi i x^2} \int_{c - i\infty}^{c + i\infty}e^{i\pi s}\log(x)^s \Gamma(s)L(s)ds + \frac{1}{2\pi i x} \int_{c - i\infty}^{c + i\infty}e^{i\pi s}\left(\frac{1}{x}s\log(x)^{s-1}\right) \Gamma(s)L(s)ds$$
$$m'(x) = \frac{1}{2\pi i x^2}\left(-\int_{c - i\infty}^{c + i\infty}e^{i\pi s}\log(x)^s \Gamma(s)L(s)ds + \int_{c - i\infty}^{c + i\infty}e^{i\pi s}s\log(x)^{s-1} \Gamma(s)L(s)ds\right)$$
$$m'(x) = \frac{1}{2\pi i x^2}\int_{c - i\infty}^{c + i\infty}\left(-e^{i\pi s}\log(x)^s \Gamma(s)L(s) + e^{i\pi s}s\log(x)^{s-1} \Gamma(s)L(s)\right)ds$$
$$m'(x) = \frac{1}{2\pi i x^2}\int_{c - i\infty}^{c + i\infty}e^{i\pi s}\left(-\log(x)^s + s\log(x)^{s-1}\right) \Gamma(s)L(s)ds$$
$$m'(x) = \frac{1}{2\pi i x^2}\int_{c - i\infty}^{c + i\infty}e^{i\pi s}\log(x)^{s - 1}(s-\log(x)) \Gamma(s)L(s)ds$$
$$M(x) = xm'(x) = \frac{1}{2\pi i x}\int_{c - i\infty}^{c + i\infty}e^{i\pi s}\log(x)^{s - 1}(s-\log(x)) \Gamma(s)L(s)ds$$



\subsection{Rambling2}
Contstruction of $m(t)$ via composition of local variants 

Let:
$$O_p : f \mapsto a_1f(t) + a_pf(t^p) + a_{p^2}f(t^{p^2}) + \ldots = \sum_{n = 0} a_{p^n}f(t^{p^n})$$
$$\hat{O}_p = O_p \circ \ldots \circ O_3 \circ O_2$$
$$m_p = \hat{O}_p(id)$$

Theorem:
$$m_{\infty}(t) = t m(t)$$

And:
$$O_p(id)(t) = Z_t(t, p)$$

\subsection{Rambling3}
Using Euler product for elliptic curve:
$$m(x) = \frac{1}{2\pi i x} \int_{c - i\infty}^{c + i\infty}e^{i\pi s}\log(x)^s \Gamma(s)L(s)ds$$
$$m(x) = \frac{1}{2\pi i x} \int_{c - i\infty}^{c + i\infty}e^{i\pi s}\log(x)^s \Gamma(s)\left(\prod_p (1 + a_p p^{-s} + p p^{-2s})^{-1}\right)ds$$


\subsection{Rambling4}
Working with a general m function from a multiplicative function $a_k$, and assuming proper convergence, we get:
$$a_n = \prod_{p_1^{e_1}p_2^{e_2}\cdots p_k^{e_k} = n} \prod_{i = 1}^k a_{p_i^{e_i}}$$
$$m(t) = \sum_{n \ge 1}t^{n - 1}a_n = \sum_{n \ge 1}t^{n - 1} \prod_{p_1^{e_1}p_2^{e_2}\cdots p_k^{e_k} = n} \prod_{i = 1}^k a_{p_i^{e_i}}$$
$$m(t) = \sum_{n \ge 1} \prod_{p_1^{e_1}p_2^{e_2}\cdots p_k^{e_k} = n} t^{(p_1^{e_1}p_2^{e_2}\cdots p_k^{e_k}) - 1} \prod_{i = 1}^k a_{p_i^{e_i}}$$
$$m(t) = \sum_{\substack{k \ge 0 \\ p_1, p_2, \ldots, p_k \text{distinct} \\ e_1, e_2, \ldots, e_k}} t^{(p_1^{e_1}p_2^{e_2}\cdots p_k^{e_k}) - 1} \prod_{i = 1}^k a_{p_i^{e_i}}$$
$$m(t) = \sum_{e_p \text{finite support}} t^{(\prod_p p^{e_p}) - 1} \prod_p a_{p^{e_p}}$$


\subsection{Rambling5}
We have the following formula for an of elliptic curves:

From wolfram alpha:
$$\frac{1}{1 + xt + yt^2} = \sum_{n = 0}t^n\left(\frac{2^{-n}\left(-x + \sqrt{x^2 - 4y}\right)^{n + 1} + \left(-\frac{x}{2} - \frac{1}{2}\sqrt{x^2 - 4y}\right)^n\left(x + \sqrt{x^2 - 4y}\right)}{2\sqrt{x^2 - 4y}}\right)$$
We compute:
$$\frac{1}{1 + xt + yt^2} = \sum_{n = 0}t^n\left(\frac{2^{-n}\left(-x + \sqrt{x^2 - 4y}\right)^{n + 1} + (-2)^{-n}\left(x + \sqrt{x^2 - 4y}\right)^n\left(x + \sqrt{x^2 - 4y}\right)}{2\sqrt{x^2 - 4y}}\right)$$
$$\frac{1}{1 + xt + yt^2} = \sum_{n = 0}t^n2^{-n - 1}\left(\frac{\left(-x + \sqrt{x^2 - 4y}\right)^{n + 1} + (-1)^n\left(x + \sqrt{x^2 - 4y}\right)^{n + 1}}{\sqrt{x^2 - 4y}}\right)$$
Thus:
$$\frac{1}{1 + a_pt + pt^2} = \sum_{n = 0}t^n2^{-n - 1}\left(\frac{\left(-a_p + \sqrt{a_p^2 - 4p}\right)^{n + 1} + (-1)^n\left(a_p + \sqrt{a_p^2 - 4p}\right)^{n + 1}}{\sqrt{a_p^2 - 4p}}\right)$$
Hence:
$$a_{p^e} = 2^{-e - 1}\left(\frac{\left(-a_p + \sqrt{a_p^2 - 4p}\right)^{e + 1} + (-1)^e\left(a_p + \sqrt{a_p^2 - 4p}\right)^{e + 1}}{\sqrt{a_p^2 - 4p}}\right)$$


\subsection{Rambling6}
$$m(x) = \frac{1}{2\pi i x} \int_{c - i\infty}^{c + i\infty}(-\log(x))^{-s} \Gamma(s)L(s)ds$$
We set $x = re^{i\phi}$
$$m(re^{i\phi}) = \frac{1}{2\pi i re^{i\phi}} \int_{c - i\infty}^{c + i\infty}(-\log(re^{i\phi}))^{-s} \Gamma(s)L(s)ds$$
$$m(re^{i\phi}) = \frac{1}{2\pi i re^{i\phi}} \int_{c - i\infty}^{c + i\infty}(-\log(r) - i\phi)^{-s} \Gamma(s)L(s)ds$$



\subsection{Rambling7}
$$m(x) = \frac{1}{2\pi i x} \int_{c - i\infty}^{c + i\infty}(-\log(x))^{-s} \Gamma(s)L(s)ds$$
We see what happens when we invert $x$:
$$m(1/x) = \frac{x}{2\pi i} \int_{c - i\infty}^{c + i\infty}(-\log(1/x))^{-s} \Gamma(s)L(s)ds$$
$$ = \frac{x}{2\pi i} \int_{c - i\infty}^{c + i\infty}\log(x)^{-s} \Gamma(s)L(s)ds$$
$$m(1/x) \frac{1}{x^2} = \frac{1}{2\pi i x} \int_{c - i\infty}^{c + i\infty}\log(x)^{-s} \Gamma(s)L(s)ds$$
Let:
$$F(x) = \left(\int_{c - i\infty}^{c + i\infty}\log(x)^{-s} \Gamma(s)L(s)ds\right)\left(\int_{c - i\infty}^{c + i\infty}(-\log(x))^{-s} \Gamma(s)L(s)ds\right)^{-1}$$
Then:
$$m(1/x) \frac{1}{x^2} = \frac{1}{2\pi i x} F(x) \int_{c - i\infty}^{c + i\infty}(-\log(x))^{-s} \Gamma(s)L(s)ds$$
$$ = F(x) m(x)$$
So we have:
$$m(1/x) = x^2 F(x) m(x)$$
What remains is to understand $F(x)$. We see:
$$F(x) = \left(\int_{c - i\infty}^{c + i\infty}\log(x)^{-s} \Gamma(s)L(s)ds\right)\left(\int_{c - i\infty}^{c + i\infty}(-1)^{-s} \log(x)^{-s} \Gamma(s)L(s)ds\right)^{-1}$$
Set:
$$u(x) = \int_{c - i\infty}^{c + i\infty}\log(x)^{-s} \Gamma(s)L(s)ds$$
$$v(x) = \int_{c - i\infty}^{c + i\infty}(-1)^{-s} \log(x)^{-s} \Gamma(s)L(s)ds$$
Then $F(x) = u(x)/v(x)$ and we have:
$$F'(x) = \frac{u'(x)v(x) - u(x)v'(x)}{v(x)^2}$$
We now compute:
$$u'(x) = \int_{c - i\infty}^{c + i\infty}(\log(x)^{-s})' \Gamma(s)L(s)ds = \int_{c - i\infty}^{c + i\infty}\frac{-s}{x} \log(x)^{-s-1} \Gamma(s)L(s)ds$$
$$ = \frac{-1}{x} \int_{c - i\infty}^{c + i\infty}\log(x)^{-(s+1)} \Gamma(s + 1)L(s)ds$$
$$v'(x) = \int_{c - i\infty}^{c + i\infty}(-1)^{-s} (\log(x)^{-s})' \Gamma(s)L(s)ds = \int_{c - i\infty}^{c + i\infty}(-1)^{-s} \frac{-s}{x} \log(x)^{-s-1} \Gamma(s)L(s)ds$$
$$ = \frac{-1}{x} \int_{c - i\infty}^{c + i\infty}(-1)^{-s}\log(x)^{-(s+1)} \Gamma(s + 1)L(s)ds$$
Let: 
$$I_1(x, s) = \log(x)^{-s} \Gamma(s) L(s)$$
$$I_2(x, s) = \log(x)^{-(s + 1)} \Gamma(s + 1) L(s)$$
Then:
$$u(x) = \int_{c - i\infty}^{c + i\infty}I_1(x, s)ds$$
$$v(x) = \int_{c - i\infty}^{c + i\infty}(-1)^{-s} I_1(x, s)ds$$
$$u'(x) = \frac{-1}{x} \int_{c - i\infty}^{c + i\infty}I_2(x, s)ds$$
$$v'(x) = \frac{-1}{x} \int_{c - i\infty}^{c + i\infty}(-1)^{-s}I_2(x, s)ds$$
We now have:
$$u(x)v'(x) = \left(\int_{c - i\infty}^{c + i\infty}I_1(x, s)ds\right)\left(\frac{-1}{x} \int_{c - i\infty}^{c + i\infty}(-1)^{-s}I_2(x, s)ds\right)$$
$$ = \frac{-1}{x} \left(\int_{c - i\infty}^{c + i\infty}I_1(x, s)ds\right)\left(\int_{c - i\infty}^{c + i\infty}(-1)^{-s}I_2(x, s)ds\right)$$
$$ = \frac{-1}{x} \left(\int_{c - i\infty}^{c + i\infty}\int_{c - i\infty}^{c + i\infty}I_1(x, s_1)(-1)^{-s_2}I_2(x, s_2)ds_1ds_2\right)$$
And we also have:
$$u'(x)v(x) = v(x)u'(x) = \left(\int_{c - i\infty}^{c + i\infty}(-1)^{-s} I_1(x, s)ds\right)\left(\frac{-1}{x} \int_{c - i\infty}^{c + i\infty}I_2(x, s)ds\right)$$
$$ = \frac{-1}{x} \left(\int_{c - i\infty}^{c + i\infty}(-1)^{-s} I_1(x, s)ds\right)\left(\int_{c - i\infty}^{c + i\infty}I_2(x, s)ds\right)$$
$$ = \frac{-1}{x} \left(\int_{c - i\infty}^{c + i\infty}\int_{c - i\infty}^{c + i\infty}(-1)^{-s_1} I_1(x, s_1)I_2(x, s_2)ds_1ds_2\right)$$
This gives us: 
$$u(x)v'(x) - u'(x)v(x)$$
$$ = \frac{-1}{x} \left(\int_{c - i\infty}^{c + i\infty}\int_{c - i\infty}^{c + i\infty}\left((-1)^{-s_2} - (-1)^{-s_1}\right)I_1(x, s_1)I_2(x, s_2)ds_1ds_2\right)$$
This is zero whenever L comes from a Dirichlet character or a Quadratic field! How?

\subsection{Rambling8}


Say we have: 
$$m(1/x) = Cx^2m(x)$$
We have:
$$(xM(x))' = m(x)$$
We insert:
$$(1/x \cdot M(1/x))' = Cx^2(xM(x))'$$
$$(-1/x^2 \cdot M(1/x) + 1/x \cdot (M(1/x))') = Cx^2(M(x) + xM'(x))$$
$$(-1/x^2 \cdot M(1/x) + 1/x \cdot (-1/x^2 M'(1/x))) = Cx^2(M(x) + xM'(x))$$
$$-1/x^2 M(1/x) + -1/x^3 M'(1/x) = Cx^2M(x) + Cx^3M'(x)$$
$$-x M(1/x) + -1 M'(1/x) = Cx^5M(x) + Cx^6M'(x)$$
This goes nowhere...

Say we have:
$$M(1/x) = Cx^2M(x)$$
$$1/xM(1/x) = CxM(x)$$
$$(1/xM(1/x))' = C(xM(x))'$$
$$m(1/x) = Cm(x)$$
Interresting!

Say we have:
$$M(1/x) = Cx^2f(x)M(x)$$
$$1/xM(1/x) = Cxf(x)M(x)$$
$$(1/xM(1/x))' = C(f(x)xM(x))'$$
$$m(1/x) = C(f'(x)xM(x) + f(x)(xM(x))')$$
$$m(1/x) = C(f'(x)xM(x) + f(x)m(x))$$

Say we have:
$$M(1/x) = Cx^2M(x) + f(x)$$
$$1/xM(1/x) = CxM(x) + 1/xf(x)$$
$$(1/xM(1/x))' = (CxM(x) + 1/xf(x))'$$
$$m(1/x) = Cm(x) + (1/xf(x))'$$
$$m(1/x) = Cm(x) + (1/xf'(x) - 1/x^2f(x))$$


\subsection{Rambling9}
Derivative of m-function formula by L:
$$m(x) = \frac{1}{2\pi i x} \int_{c - i\infty}^{c + i\infty}(-\log(x))^{-s} \Gamma(s)L(s)ds$$
$$m'(x) = \left(\frac{1}{2\pi i x}\right)' \int_{c - i\infty}^{c + i\infty}(-\log(x))^{-s} \Gamma(s)L(s)ds + \frac{1}{2\pi i x} \left(\int_{c - i\infty}^{c + i\infty}(-\log(x))^{-s} \Gamma(s)L(s)ds\right)'$$
$$m'(x) = \frac{-1}{2\pi i x^2} \int_{c - i\infty}^{c + i\infty}(-\log(x))^{-s} \Gamma(s)L(s)ds + \frac{1}{2\pi i x} \left(\int_{c - i\infty}^{c + i\infty}(-\log(x))^{-s} \Gamma(s)L(s)ds\right)'$$
$$m'(x) = \frac{-1}{2\pi i x^2} \int_{c - i\infty}^{c + i\infty}(-\log(x))^{-s} \Gamma(s)L(s)ds + \frac{1}{2\pi i x} \int_{c - i\infty}^{c + i\infty}\left((-\log(x))^{-s}\right)' \Gamma(s)L(s)ds$$
$$m'(x) = \frac{-1}{2\pi i x^2} \int_{c - i\infty}^{c + i\infty}(-\log(x))^{-s} \Gamma(s)L(s)ds + \frac{1}{2\pi i x} \int_{c - i\infty}^{c + i\infty}\frac{s}{x}(-\log(x))^{-(s + 1)} \Gamma(s)L(s)ds$$
$$m'(x) = \frac{-1}{2\pi i x^2} \int_{c - i\infty}^{c + i\infty}(-\log(x))^{-s} \Gamma(s)L(s)ds + \frac{1}{2\pi i x^2} \int_{c - i\infty}^{c + i\infty}(-\log(x))^{-(s + 1)} \Gamma(s + 1)L(s)ds$$
As $L(s - 1)$ corresponds to multiplication by $1/n$, we see that:
$$m^{(-1)}(x) = \frac{1}{2\pi i x} \int_{c - i\infty}^{c + i\infty}(-\log(x))^{-s} \Gamma(s)L(s - 1)ds$$
$$ = \frac{1}{2\pi i x} \int_{c - i\infty}^{c + i\infty}(-\log(x))^{-(s + 1)} \Gamma(s + 1)L(s)ds$$
Because $c$ doesn't matter if it is big enough. We then get:
$$m'(x) = \frac{-1}{x} m(x) + \frac{1}{x} m^{(-1)}(x)$$
$$xm'(x) = -m(x) + m^{(-1)}(x)$$
$$M(x) = -m(x) + m^{(-1)}(x)$$
Interresting!

We begin with $M(x)$:
$$M(x) = \frac{1}{2\pi i x} \int_{c - i\infty}^{c + i\infty}(-\log(x))^{-s} \Gamma(s)L(s + 1)ds$$
$$M'(x) = \left(\frac{1}{2\pi i x}\right)' \int_{c - i\infty}^{c + i\infty}(-\log(x))^{-s} \Gamma(s)L(s + 1)ds + \frac{1}{2\pi i x} \left(\int_{c - i\infty}^{c + i\infty}(-\log(x))^{-s} \Gamma(s)L(s + 1)ds\right)'$$
$$M'(x) = \frac{-1}{2\pi i x^2} \int_{c - i\infty}^{c + i\infty}(-\log(x))^{-s} \Gamma(s)L(s + 1)ds + \frac{1}{2\pi i x} \left(\int_{c - i\infty}^{c + i\infty}(-\log(x))^{-s} \Gamma(s)L(s + 1)ds\right)'$$
$$M'(x) = \frac{-1}{2\pi i x^2} \int_{c - i\infty}^{c + i\infty}(-\log(x))^{-s} \Gamma(s)L(s + 1)ds + \frac{1}{2\pi i x} \int_{c - i\infty}^{c + i\infty}\left((-\log(x))^{-s}\right)' \Gamma(s)L(s + 1)ds$$
$$M'(x) = \frac{-1}{2\pi i x^2} \int_{c - i\infty}^{c + i\infty}(-\log(x))^{-s} \Gamma(s)L(s + 1)ds + \frac{1}{2\pi i x} \int_{c - i\infty}^{c + i\infty}\frac{s}{x}(-\log(x))^{-(s + 1)} \Gamma(s)L(s + 1)ds$$
$$M'(x) = \frac{-1}{2\pi i x^2} \int_{c - i\infty}^{c + i\infty}(-\log(x))^{-s} \Gamma(s)L(s + 1)ds + \frac{1}{2\pi i x^2} \int_{c - i\infty}^{c + i\infty}(-\log(x))^{-(s + 1)} \Gamma(s + 1)L(s + 1)ds$$
Because $c$ doesn't matter, we get:
$$M'(x) = \frac{-1}{2\pi i x^2} \int_{c - i\infty}^{c + i\infty}(-\log(x))^{-s} \Gamma(s)L(s + 1)ds + \frac{1}{2\pi i x^2} \int_{c - i\infty}^{c + i\infty}(-\log(x))^{-s} \Gamma(s)L(s)ds$$
We get:
$$M'(x) = -\frac{1}{x} M(x) + \frac{1}{x}m(x)$$
We have that: 
$$(xM(x))' = m(x)$$
$$M(x) + xM'(x) = m(x)$$
We set in:
$$M(x) + x\left(-\frac{1}{x} M(x) + \frac{1}{x}m(x)\right) = m(x)$$
$$M(x) + (-M(x) + m(x)) = m(x)$$
$$m(x) = m(x)$$
Nothing new...

\subsection{Rambling10}

Let:
$$m : \C[\epsilon] \to \C[\epsilon]$$
$$z \mapsto 1 + a_2z + \ldots$$
We see: 
$$m(x + y\epsilon) = \sum_{n = 1} a_n (x + y\epsilon)^{n - 1}$$
$$ = \sum_{n = 1} a_n (x^{n - 1} + (n - 1)x^{n - 2}y\epsilon)$$
$$ = \sum_{n = 1} a_n x^{n - 1} + \sum_{n = 1} a_n (n - 1)x^{n - 2}y\epsilon$$
$$ = m(x) + y\epsilon(\sum_{n = 1} a_n (n - 1)x^{n - 2})$$
$$ = m(x) + \frac{1}{x}y\epsilon(\sum_{n = 1} a_n (n - 1)x^{n - 1})$$
$$ = m(x) + \frac{1}{x}y\epsilon(\sum_{n = 1} a_n n x^{n - 1} - \sum_{n = 1} a_n x^{n - 1})$$
$$ = m(x) + \frac{1}{x}y\epsilon(\sum_{n = 1} a_n n x^{n - 1} - m(x))$$
Let $m^{(1)}$ be such that $(xm(x))' = m^{(1)}(x)$
$$ = m(x) + \frac{y}{x}\left(m^{(1)}(x) - m(x)\right)\epsilon$$

$$m(0 + y\epsilon) = 1 + a_2y\epsilon$$

\subsection{Rambling11}
In this section we look at the $p$-adic world. Assume $a_n \in \Q$. Define:
$$m : \Q_p \to \Q_p$$
In the usual way. We see that we only have convergence when $|x|_p < 1$. 

Computation gives:

$\ldots103031422022141$ is $5$-adic zero for 389.a1

What does this mean when it isn't really well defined? We need $|x|_p < 1$

Let's work:

If $|x|_p < 1$ then: 
$$x = p(\sum_{k = 0} x_k p^k)$$
$$m(x) = m(p(\sum_{k = 0} x_k p^k)) = \sum_{n = 1} a_n (p(\sum_{k = 0} x_k p^k))^{n - 1}$$
$$ = \sum_{n = 1} a_n p^{n - 1}\left(\sum_{k = 0} x_k p^k\right)^{n - 1}$$
$$ = \sum_{n = 1} a_n p^{n - 1}\left(\sum_{k_1 + k_2 + \ldots + k_m = n - 1} \binom{n - 1}{k_1, k_2, \ldots, k_m} \prod_{i = 1}^m (x_ip^i)^{k_i} \right)$$
$$ = \sum_{n = 1} \sum_{k_1 + k_2 + \ldots + k_m = n - 1} a_{k_1 + \ldots + k_m + 1} p^{k_1 + \ldots + k_m} \binom{k_1 + \ldots + k_m}{k_1, k_2, \ldots, k_m} \prod_{i = 1}^m (x_ip^i)^{k_i}$$
$$ = \sum_{k_1, k_2, \ldots, k_m} a_{k_1 + \ldots + k_m + 1} p^{k_1 + \ldots + k_m} \binom{k_1 + \ldots + k_m}{k_1, k_2, \ldots, k_m} \prod_{i = 1}^m (x_ip^i)^{k_i}$$
$$ = \sum_{k_1, k_2, \ldots, k_m} a_{k_1 + \ldots + k_m + 1} p^{k_1 + \ldots + k_m} \binom{k_1 + \ldots + k_m}{k_1, k_2, \ldots, k_m} \prod_{i = 1}^m x_i^{k_i}p^{ik_i}$$
$$ = \sum_{k_1, k_2, \ldots, k_m} a_{k_1 + \ldots + k_m + 1} p^{k_1 + 2k_2 + \ldots + mk_m} p^{k_1 + \ldots + k_m} \binom{k_1 + \ldots + k_m}{k_1, k_2, \ldots, k_m}  \prod_{i = 1}^m x_i^{k_i}$$
$$ = \sum_{k_1, k_2, \ldots, k_m} a_{k_1 + \ldots + k_m + 1} p^{k_1 + 2k_2 + \ldots + mk_m} p^{k_1 + \ldots + k_m} \binom{k_1 + \ldots + k_m}{k_1, k_2, \ldots, k_m}  \prod_{i = 1}^m x_i^{k_i}$$
$$ = \sum_{k_1, k_2, \ldots, k_m} p^{2k_1 + 3k_2 + \ldots + (m + 1)k_m} \binom{k_1 + \ldots + k_m}{k_1, k_2, \ldots, k_m} a_{k_1 + \ldots + k_m + 1} \prod_{i = 1}^m x_i^{k_i}$$
$$ = \sum_{n = 0} \sum_{2k_1 + 3k_2 + \ldots + (m + 1)k_m = n} p^{2k_1 + 3k_2 + \ldots + (m + 1)k_m} \binom{k_1 + \ldots + k_m}{k_1, k_2, \ldots, k_m} a_{k_1 + \ldots + k_m + 1} \prod_{i = 1}^m x_i^{k_i}$$
$$ = \sum_{n = 0} p^n \sum_{2k_1 + 3k_2 + \ldots + (m + 1)k_m = n} \binom{k_1 + \ldots + k_m}{k_1, k_2, \ldots, k_m} a_{k_1 + \ldots + k_m + 1} \prod_{i = 1}^m x_i^{k_i}$$

Is this useful?
It might even be wrong because of $a_i^{k_i} > p$



\subsection{Rambling12}
In this section we define something similar to an m-function to attempt to get magical zeroes. Basically, define m so that we can do the integral-trick, and then it should have a zero at $\exp(-\frac{2\pi}{\sqrt{N}})$


\subsection{Rambling13}
I may have found a reason why $M(E(389)) = 0$! Becuase:
$$\Lambda(2 - s) = \Lambda(s)$$
$$\Lambda(2 - s - 1) = \Lambda(s - 1)$$
$$\Lambda(1 - s) = \Lambda(s - 1)$$
$$\Lambda(s) = \Lambda(-s)$$
Now when we integrate we get cancellation!

How is it related to the rank? The rank is the behavior at the critical point...

\subsection{Rambling14}

Say we have:
$$m(1/x) = F(x)$$
$$\int m(1/x) dx = \int F(x) dx$$
$$1/xM(1/x) = \int F(x) dx$$
$$M(1/x) = x \int F(x) dx$$

Let $F(x) = Cx^2m(x)$. Then:
$$M(1/x) = x \int Cx^2m(x) dx$$
$$M(1/x) = Cx \int x^2m(x) dx$$
$$M(1/x) = Cx (\int x^2(xM(x))' dx)$$
$$M(1/x) = Cx (x^2(xM(x)) - \int 2x(xM(x)) dx)$$
$$M(1/x) = Cx (x^3M(x) - 2 \int x^2M(x) dx)$$

\subsection{Rambling15}
$$m(x) = \frac{1}{2\pi i x} \int_{c - i\infty}^{c + i\infty}(-\log(x))^{-s} \Gamma(s)L(s)ds$$
$$M(x) = \frac{1}{2\pi i x} \int_{c - i\infty}^{c + i\infty}(-\log(x))^{-s} \Gamma(s)L(s - 1)ds$$
$$M(x) = \frac{1}{2\pi i x} \int_{c - i\infty}^{c + i\infty}(-\log(x))^{-1}(-\log(x))^{-(s - 1)} (s - 1)\Gamma(s - 1)L(s - 1)ds$$
$$M(x) = \frac{(-\log(x))^{-1}}{2\pi i x} \int_{c - i\infty}^{c + i\infty}(-\log(x))^{-(s - 1)} (s - 1)\Gamma(s - 1)L(s - 1)ds$$
$$M(x) = \frac{(-\log(x))^{-1}}{2\pi i x} \left(\int_{c - i\infty}^{c + i\infty}(-\log(x))^{-(s - 1)}s\Gamma(s - 1)L(s - 1)ds - \int_{c - i\infty}^{c + i\infty}(-\log(x))^{-(s - 1)}\Gamma(s - 1)L(s - 1)ds\right)$$
$$M(x) = \frac{(-\log(x))^{-1}}{2\pi i x} \left(\int_{c - i\infty}^{c + i\infty}(-\log(x))^{-(s - 1)}s\Gamma(s - 1)L(s - 1)ds - m(x)\right)$$
$$M(x) = \frac{(-\log(x))^{-1}}{2\pi i x} \left(\int_{c - i\infty}^{c + i\infty}(-\log(x))^{-(s - 1)}s\Gamma(s - 1)L(s - 1)ds\right) - \frac{(-\log(x))^{-1}}{2\pi i x}m(x)$$
$$-2\pi ix\log(x)M(x) = \int_{c - i\infty}^{c + i\infty}(-\log(x))^{-(s - 1)}s\Gamma(s - 1)L(s - 1)ds - m(x)$$
$$m(x) = 2\pi ix\log(x)M(x) + \int_{c - i\infty}^{c + i\infty}(-\log(x))^{-(s - 1)}s\Gamma(s - 1)L(s - 1)ds$$
$$m(x) = 2\pi ix\log(x)M(x) + \int_{c - i\infty}^{c + i\infty}(-\log(x))^{-s}(s + 1)\Gamma(s)L(s)ds$$
$$m(x) = 2\pi ix\log(x)M(x) + \int_{c - i\infty}^{c + i\infty}(-\log(x))^{-s}s\Gamma(s)L(s)ds + \int_{c - i\infty}^{c + i\infty}(-\log(x))^{-s}\Gamma(s)L(s)ds$$
$$m(x) = 2\pi ix\log(x)M(x) + \int_{c - i\infty}^{c + i\infty}(-\log(x))^{-s}s\Gamma(s)L(s)ds + m(x)$$
$$-2\pi ix\log(x)M(x)= \int_{c - i\infty}^{c + i\infty}(-\log(x))^{-s}s\Gamma(s)L(s)ds$$
$$M(x) = \frac{-1}{2\pi ix\log(x)}\int_{c - i\infty}^{c + i\infty}(-\log(x))^{-s}s\Gamma(s)L(s)ds$$

Can this be used to prove $M(E(N)) = 0$??! Let's try:
$$M(e^{-\frac{2\pi}{\sqrt{N}}}) = \frac{-1}{2\pi ie^{-\frac{2\pi}{\sqrt{N}}}\log(e^{-\frac{2\pi}{\sqrt{N}}})}\int_{c - i\infty}^{c + i\infty}(-\log(e^{-\frac{2\pi}{\sqrt{N}}}))^{-s}s\Gamma(s)L(s)ds$$
$$M(e^{-\frac{2\pi}{\sqrt{N}}}) = \frac{-1}{2\pi ie^{-\frac{2\pi}{\sqrt{N}}}\log(e^{-\frac{2\pi}{\sqrt{N}}})}\int_{c - i\infty}^{c + i\infty}(-\log(e^{-\frac{2\pi}{\sqrt{N}}}))^{-s}s\Gamma(s)L(s)ds$$

Let's focus on the integral:
$$\int_{c - i\infty}^{c + i\infty}(-\log(e^{-\frac{2\pi}{\sqrt{N}}}))^{-s}s\Gamma(s)L(s)ds$$
$$ = \int_{c - i\infty}^{c + i\infty}\left(\frac{2\pi}{\sqrt{N}}\right)^{-s}s\Gamma(s)L(s)ds$$

We know that:
$$\Lambda(s) = N^{s/2} \Gamma_\C(s) L(s)$$
$$ = \Lambda(2 - s)$$
We see: 
$$\Lambda(s) = N^{s/2} 2 (2\pi)^{-s} \Gamma(s) L(s)$$
$$ = 2 \sqrt{N}^{s} (2\pi)^{-s} \Gamma(s) L(s)$$
$$ = 2 \left(\frac{2\pi}{\sqrt{N}}\right)^{-s} \Gamma(s) L(s)$$
So we see:
$$\int_{c - i\infty}^{c + i\infty}\left(\frac{2\pi}{\sqrt{N}}\right)^{-s}s\Gamma(s)L(s)ds$$
$$ = \int_{c - i\infty}^{c + i\infty}\frac{s}{2} \Lambda(s)ds$$
$$ = \int_{1 - i\infty}^{1 + i\infty}\frac{s}{2} \Lambda(s)ds$$
$$ = i \int_{-\infty}^{\infty}\frac{1 + ui}{2} \Lambda(1 + ui)du$$
$$ = i \int_{-\infty}^{\infty}\frac{1 + ui}{2} \Lambda(2 - (1 + ui))du$$
$$ = i \int_{-\infty}^{\infty}\frac{1 + ui}{2} \Lambda(1 - ui)du$$
$$ = -i \int_{-\infty}^{\infty}\frac{1 - ui}{2} \Lambda(1 + ui)du$$
We have:
$$i \int_{-\infty}^{\infty}\frac{1 + ui}{2} \Lambda(1 + ui)du = -i \int_{-\infty}^{\infty}\frac{1 - ui}{2} \Lambda(1 + ui)du$$
$$ \int_{-\infty}^{\infty}\frac{1 + ui}{2} \Lambda(1 + ui)du + \int_{-\infty}^{\infty}\frac{1 - ui}{2} \Lambda(1 + ui)du= 0$$
$$ \int_{-\infty}^{\infty}\frac{1 + ui}{2} \Lambda(1 + ui) + \frac{1 - ui}{2} \Lambda(1 + ui)du = 0$$
$$ \int_{-\infty}^{\infty}\Lambda(1 + ui)du = 0$$


\subsection{Rambling16}
Here we will look at modular forms:
$$f(z) = \sum_{n = 1} a_n e^{2i\pi nz}$$
$$ = \sum_{n = 1} a_n q^n$$
$$ = q \sum_{n = 1} a_n q^{n - 1}$$
$$ = q m(q) = e^{2i\pi z} m(e^{2i\pi z})$$

We know that $f(z + 1) = f(z)$ and $f(-1/z) = z^kf(z)$. The first seems to be completely vacous. The second:
$$f(-1/z) = z^kf(z)$$
$$e^{2i\pi (-1/z)} m\left(e^{2i\pi (-1/z)}\right) = z^k(e^{2i\pi z} m(e^{2i\pi z}))$$
$$\frac{1}{e^{2i\pi (1/z)}} m\left(\frac{1}{e^{2i\pi (1/z)}}\right) = z^k e^{2i\pi z} m(e^{2i\pi z})$$
$$m\left(\frac{1}{e^{2i\pi (1/z)}}\right) = z^k e^{2i\pi (1/z)} e^{2i\pi z} m(e^{2i\pi z})$$
$$m\left(e^{2i\pi (-1/z)}\right) = z^k e^{2i\pi (z + 1/z)} m(e^{2i\pi z})$$
$$m\left(e^{2i\pi (-1/z)}\right) = z^k e^{2i\pi z} e^{2i\pi 1/z} m(e^{2i\pi z})$$
Let:
$$x = e^{2i\pi (-1/z)}$$
Then:
$$\log(x) = 2i\pi (-1/z)$$
$$z = \frac{-2i\pi}{\log(x)}$$
We get:
$$m(x) = \left(\frac{-2i\pi}{\log(x)}\right)^k e^{2i\pi \frac{-2i\pi}{\log(x)}} \frac{1}{x} m\left(e^{2i\pi \frac{-2i\pi}{\log(x)}}\right)$$
$$m(x) = \left(\frac{-2i\pi}{\log(x)}\right)^k e^{2i\pi \frac{-2i\pi}{\log(x)}} \frac{1}{x} m\left(e^{2i\pi \frac{-2i\pi}{\log(x)}}\right)$$
We see:
$$e^{2i\pi \frac{-2i\pi}{\log(x)}}$$
$$ = e^{4\pi^2 \log(x)^{-1}}$$
$$ = \exp(4\pi^2 \log(x)^{-1})$$
We get:
$$m(x) = \left(\frac{-2i\pi}{\log(x)}\right)^k \exp(4\pi^2 \log(x)^{-1}) \frac{1}{x} m\left(\exp(4\pi^2 \log(x)^{-1})\right)$$
$$m(x) = \frac{1}{x} \left(\frac{-2i\pi}{\log(x)}\right)^k \exp(4\pi^2 \log(x)^{-1}) m\left(\exp(4\pi^2 \log(x)^{-1})\right)$$
How can we understand this strange transform?
Let $T : x \to \exp(4\pi^2 \log(x)^{-1})$
Say $x = re^{i\phi}$:
$$T(x) = \exp(4\pi^2 \log(re^{i\phi})^{-1})$$
$$T(x) = \exp(4\pi^2 (\log(r) + i\phi)^{-1})$$
$$T(x) = \exp(4\pi^2 \frac{\log(r) - i\phi}{\log(r)^2 + \phi^2})$$
$$T(x) = \exp(4\pi^2 \frac{\log(r)}{\log(r)^2 + \phi^2} + 4\pi^2 \frac{-i\phi}{\log(r)^2 + \phi^2})$$
$$T(x) = \exp(\frac{\log(r)4\pi^2}{\log(r)^2 + \phi^2}) \exp(i\frac{-\phi4\pi^2}{\log(r)^2 + \phi^2})$$

Tests:
Let $r = 1$:
$$T(x) = \exp(\frac{0\cdot 4\pi^2}{\phi^2}) \exp(i\frac{-\phi4\pi^2}{\log(r)^2 + \phi^2})$$
$$T(x) = \exp(i\frac{-\phi4\pi^2}{\log(r)^2 + \phi^2})$$

\subsection{Rambling16}
$$m(x) = \frac{1}{2\pi i x} \int_{c - i\infty}^{c + i\infty}(-\log(x))^{-s} \Gamma(s)L(s)ds$$
Let $x = e^{-y\frac{2\pi}{\sqrt{N}}}$
$$m(x) = \frac{1}{2\pi i x} \int_{c - i\infty}^{c + i\infty} \left(y\frac{2\pi}{\sqrt{N}}\right)^{-s} \Gamma(s)L(s)ds$$
$$m(x) = \frac{1}{2\pi i x} \int_{c - i\infty}^{c + i\infty} y^{-s}\left(\frac{2\pi}{\sqrt{N}}\right)^{-s} \Gamma(s)L(s)ds$$
$$m(x) = \frac{1}{2\pi i x} \int_{c - i\infty}^{c + i\infty} y^{-s}\frac{1}{2}\Lambda(s)ds$$

We can create this for anything with a functional equation!


\subsection{Rambling17}
Let $a_n$ have a bell-antiderivative: 
$$X = p \mapsto \sum_{i = 0}^{k_p} w_{p, i}[\alpha_{p, i}]$$
Then:
$$a_{p^e} = \upsilon_p(\psi^e (X)) = \sum_{i = 0}^{k_p} w_{p, i}\alpha_{p, i}^e$$
Then: 
$$F(t) = \sum_{n = 0}^\infty a_{p^n}t^n = \sum_{n = 0}^\infty \sum_{i = 0}^{k_p} w_{p, i}\alpha_{p, i}^n t^n $$
$$ = \sum_{i = 0}^{k_p} w_{p, i} \sum_{n = 0}^\infty (\alpha_{p, i}t)^n $$
$$ = \sum_{i = 0}^{k_p} \frac{w_{p, i}}{1 - \alpha_{p, i}t} $$
This is important!


\subsection{Rambling18}
Let's try this for $m$-function: 
$$m(t) = \sum_{e_p \text{finite support}} t^{(\prod_p p^{e_p}) - 1} \prod_p a_{p^{e_p}}$$
$$m(t) = \sum_{e_p \text{finite support}} t^{(\prod_p p^{e_p}) - 1} \prod_p \sum_{i = 0}^{k_p} w_{p, i}\alpha_{p, i}^{e_p}$$
$$m(t) = \frac{1}{t}\sum_{e_p \text{finite support}} t^{\prod_p p^{e_p}} \prod_p \sum_{i = 0}^{k_p} w_{p, i}\alpha_{p, i}^{e_p}$$


\subsection{Rambling19}
From before:
Contstruction of $m(t)$ via composition of local variants 

Let:
$$O_p : f \mapsto a_1f(t) + a_pf(t^p) + a_{p^2}f(t^{p^2}) + \ldots = \sum_{n = 0} a_{p^n}f(t^{p^n})$$
$$\hat{O}_p = O_p \circ \ldots \circ O_3 \circ O_2$$
$$m_p = \hat{O}_p(id)$$

Theorem:
$$m_{\infty}(t) = t m(t)$$

And:
$$O_p(id)(t) = Z_t(t, p)$$

We now see:
$$O_p : f \mapsto \sum_{n = 0} a_{p^n}f(t^{p^n}) = \sum_{n = 0} \sum_{i = 0}^{k_p} w_{p, i}\alpha_{p, i}^nf(t^{p^n})$$
$$ = \sum_{i = 0}^{k_p} w_{p, i} \sum_{n = 0} \alpha_{p, i}^nf(t^{p^n})$$

\subsection{Rambling20}
\begin{verbatim}
  
Definer:
m_p(z; a_n) = sum_n a_(p * n) z^(p*n)
Dette er en slags lokal m-funksjon.

Observerer følgende sum:
sum_p m_p(z; a_n) = sum_p sum_n a_(p*n) z^(p*n) = sum_{p*n = i} a_i z^i = sum_i omega(i) a_i z^i = m(z; omega boxprod a_n)
Altså er det å summere over prekomposisjoner av m-funksjonen med alle primtallseksponenter det samme som boxprodukt med omega, antall distinkte primtall i et tall. 

Videre merk også at m_p kan deles opp for å bruke multiplikativitet:
m_p(z; a_n) = sum_{e, n coprime with p, p^e*n = m} a_(p * p^e*n ) z^(p*p^e*n) = sum_e (a_p^(e + 1)) sum_{n coprime p} a_n (z^(p^(e + 1)))^n
Sett nå C_p(z; a_n) := sum_{n coprime p} a_n z^n
Da får vi:
m_p(z; a_n) =  sum_e (a_(p^(e + 1))) C_p(z^(p^(e + 1))) = sum_(e = 1...) a_(p^e) C_p(z^(p^e)) = sum_(e = 1...) a_(p^e) C_p(z^(p^e))

Merk nå at C_p(z) = m(z) - \frac{1}{p}\sum_{0 \le i < p} m(\omega^i z). Dette skrev jeg om på github.
Vi kan altså utrykke m_p(z; a_n) utelukkende ved å bruke m-funksjonen og a_p^e. 

Merk også at det er åpenbart at m_p(z; a_n) + C_p(z; a_n) = m(z; a_n), og dette gir også:
m_p(z; a_n) = \frac{1}{p}\sum_{0 \le i < p} m(\omega^i z) = 1/p sum_{i < p} m(omega^i z)

Altså:
m(z; omega boxprod a_n) = sum_p m_p(z; a_n) = sum_p 1/p sum_{i < p} m(omega^i z) 
Og:
1/p sum_{i < p} m(omega^i z) = sum_(e = 1...) a_(p^e) C_p(z^(p^e)) = sum_(e = 1...) a_(p^e) (m(z^(p^e)) - 1/p sum_{i < p} m(\omega^i z^(p^e)))

Forøvrig også:
m_p(z; a_n) = sum_(e = 1...) a_(p^e) (m(z^(p^e)) - m_p(z^(p^e)))

\end{verbatim}


\subsection{Rambling21}
Let's try to find FE for completely multiplicative:
$$m(t) = \sum_{e_p \text{finite support}} t^{(\prod_p p^{e_p}) - 1} \prod_p a_{p^{e_p}}$$
$$ = \sum_{e_p \text{finite support}} t^{(\prod_p p^{e_p}) - 1} \prod_p a_p^{p_e}$$
$$ = \frac{1}{t} \sum_{e_p \text{finite support}} t^{\prod_p p^{e_p}} \prod_p a_p^{p_e}$$
$$m(1/t) = t \sum_{e_p \text{finite support}} t^{-\prod_p p^{e_p}} \prod_p a_p^{p_e}$$


\subsection{Rambling21}
We have $b_n$ from $a_n$ bell derivative, which means $b_n = (a_n \boxtimes \tau) \ominus a_n$ or $b_n \oplus a_n = a_n \boxtimes \tau$. We get:
$$m(z; b_n \oplus a_n) = m(z; a_n \boxtimes \tau)$$

We know that $\tau$ is the number of divisors. So for each $n$ define:
$$m_n(z) = \sum_{i = 1} a_{in} z^{in - 1}$$
We see: 
$$\sum_{n = 1} m_n(z) = \sum_{n = 1}\sum_{i = 1} a_{in} z^{in - 1}$$
$$ = \sum_{n = 1} \tau(n) a_{n} z^{n - 1}$$
$$ = m(z; a_n \boxtimes \tau)$$
We see: 
$$m(z; b_n \oplus a_n) = \sum_n a_n m(z^n; b_n) = \sum_n b_n m(z^n; a_n)$$


\subsection{Rambling22}
Let $b_n$ be some transform of $a_n$. We see: 
$$m(t) = \sum_n a_n t^{n - 1} = \frac{1}{t}\sum_n a_n t^n = \frac{1}{t}\sum_n b_n \sum_{i = 1} t^{in}$$
$$ = \frac{1}{t}\sum_n b_n \frac{t^n}{1 - t^n}$$
$$m(t^{-1}) = t \sum_n b_n \frac{t^{-n}}{1 - t^{-n}}$$
$$m(t^{-1}) = t \sum_n b_n \frac{1}{t^{n} - 1}$$
Another direction:
$$m(t) = \frac{1}{t}\sum_n b_n \frac{t^n}{1 - t^n}$$
$$ = \frac{1}{t}\sum_n b_n \frac{-(1 - t^n) + 1}{1 - t^n}$$
$$ = \frac{1}{t}\sum_n b_n (\frac{1}{1 - t^n} + \frac{-(1 - t^n)}{1 - t^n})$$
$$ = \frac{1}{t}\sum_n b_n (\frac{1}{1 - t^n} - 1)$$
$$ = \frac{1}{t}\left(\sum_n \frac{b_n}{1 - t^n} - \sum_n b_n \right)$$
How do we get $b_n$ to be summable? One way is if it is finite!
Let: 
$$F(t) = \sum_n \frac{b_n}{1 - t^n}$$
We see: 
$$m(t) = \frac{1}{t}\left(F(t) - F(0)\right)$$
$$ = \frac{1}{t}\left(\int_0^t F'(x) dx\right)$$
Let $f(t) = F'(t)$. We see: 
$$f(t) = \sum_n b_n \frac{nt^{n - 1}}{(1 - t^n)^2}$$
Which gives:
$$m(t) = \frac{1}{t}\left(\int_0^t \sum_n b_n \frac{nx^{n - 1}}{(1 - x^n)^2} dx \right)$$



\subsection{Rambling23}
Okay, $M$ or $m^{(-2)}$ can actually converge for $|z| = 1$, when $a_n$ goes up and down. We can use to explicitly compute $m(1/z)/m(z)$


\subsection{Rambling24}
Let: 
$$me(t) = \sum_{n = 1} \frac{a_n}{(n - 1)!}t^{n - 1}$$
When $a_n = 1$ we get $\exp$. One dirichlet character is $\cos$ and another is $\cosh$ I think. We see that for $a_n$ periodic we get linear differential equations, namely $me^{(n)} = me$. Let's try to rewrite:
$$me(t) = \sum_{n = 0} \frac{b'_n}{n!}t^{n}$$
$$me(t) = 1 + \sum_{n = 1} b_n \sum_{i = 1} \frac{1}{(in)!}t^{in}$$
$$me(t) = 1 + \sum_{n = 1} b_n / n \left(-1 + \sum_{k = 1}^n \exp(e^{2\pi ik/n}t)\right)$$
$$me(t) = 1 - \sum_{n = 1} b_n / n + \sum_{n = 1} b_n / n \sum_{k = 1}^n \exp(e^{2\pi ik/n}t)$$
$$me(t) = 1 - \sum_{n = 1} b_n / n + \sum_{n = 1} b_n / n \sum_{k = 1}^n \exp(e^{2\pi ik/n}t)$$


Numerically, we find three zeroes and $z_0 = z_1 \log(z_1)$

\subsection{Rambling25}
Very simple idea: Taylor-develop m-function at $1/2$. 

\subsection{Rambling26}
We have:
$$m(z) = \int_0^\infty me(tz) e^{-t} dt$$
$$me(z) = \frac{1}{2\pi} \int_{-\pi}^{\pi} m(ze^{-i\theta}) \exp(\exp(i\theta)) d\theta$$

\subsection{Rambling27}

We want to study $m$ for completely multiplicative $a_n$. Let: 
$$$$


\subsection{Rambling28}

We decompose $X$ into $X_p$ where $X = \bigoplus_p X_p$ and $X_p$ is zero everywhere but for $p^e$. We get $a_1 = 1$ and $a_{p^e}$ has values and everywhere else $a_n = 0$. We get:

Well, just rambling 2...

Can we obtain a FE?

\subsection{Rambling29}
This is wrong formula for $m_{\ne p}$!
We remove $p^e$ and then add it back:
$$m_{\ne p}(z) = \frac{1}{p} \sum_{k = 1}^p m(z \omega^k)$$

$$m(z) = \sum_{n = 0} a_{p^n} m_{\ne p}(z^{p^n})$$
$$ = \sum_{n = 0} a_{p^n} \frac{1}{p} \sum_{k = 1}^p m(z^{p^n} \omega^k)$$
$$ = \sum_{k = 1}^p \sum_{n = 0} a_{p^n} \frac{1}{p} m(z^{p^n} \omega^k)$$


\subsection{Rambling30}
Let $\omega = e^{2\pi i/p}$, assume $|z| < 1$ and let:
$$m_{\ne p}(z) = \frac{1}{p} \sum_{k = 1}^p m(z \omega^k)$$
We see: 
$$\lim_{p \to \infty} m_{\ne p}(z) = \int_{0}^{1} m(z e^{\phi i 2\pi}) d\phi$$
Let $x = ze^{\phi i 2\pi}$ and $C$ be this contour starting at $z$. Then $dx/d\phi = 2\pi ize^{\phi i 2\pi}$ and $d\phi = dx \frac{1}{2\pi ize^{\phi i 2\pi}}$. We get:
$$\int_{0}^{1} m(z e^{\phi i 2\pi}) d\phi = \int_C \frac{1}{2\pi ize^{\phi i 2\pi}} m(x) dx = \frac{1}{2\pi i}\int_C \frac{1}{x} m(x) dx = \text{Res}_{x = 0} (\frac{m(x)}{x}) = m(0) = 1$$
We get:
$$\lim_{p \to \infty} m_{\ne p}(z) = 1$$
HOWEVER!!!
$$\lim_{p \to \infty} m_{\ne p}(z) = m(z)$$
Becuase we are remove more and more insignificant terms!

We need to examine this. 



\subsection{Rambling31}
Let $\omega = e^{2\pi i/p}$, assume $|z| < 1$ and let:
$$m_{\ne p}(z) = \frac{1}{p} \sum_{k = 1}^p m(z \omega^k)$$

We see: 
$$m_{\ne p}(z) = \frac{1}{p} \sum_{k = 1}^p \sum_{n = 1} a_n (z \omega^k)^{n - 1}$$
$$ = \sum_{n = 1} a_n \frac{1}{p} \sum_{k = 1}^p (z \omega^k)^{n - 1}$$
$$ = \sum_{n = 1} a_n \frac{1}{p} \sum_{k = 1}^p z^{n - 1} \omega^{k(n - 1)}$$
$$ = \sum_{n = 1} a_n z^{n - 1} \frac{1}{p} \sum_{k = 1}^p \omega^{k(n - 1) \text{ mod } p}$$
We examine:
$$\frac{1}{p} \sum_{k = 1}^p \omega^{k(n - 1) \text{ mod } p}$$
Case 1, $n - 1 \text{ mod } p = 0$:
$$\frac{1}{p} \sum_{k = 1}^p \omega^{k(n - 1) \text{ mod } p} = \frac{1}{p} \sum_{k = 1}^p 1^{k} = 1$$
Case 2, $n - 1 \text{ mod } p \ne 0$:
We get balance, and we get $0$.

This is not what we want!


\subsection{Rambling32}
Let's try to fix this:
$$z m_{\ne p}(z) = \frac{1}{p} \sum_{k = 1}^p z \omega^k m(z \omega^k)$$
$$m_{\ne p}(z) = \frac{1}{p} \sum_{k = 1}^p \omega^k m(z \omega^k)$$



\subsection{Rambling33}
We remove $p^e$ and then add it back:
$$m_{\ne p}(z) = \frac{1}{p} \sum_{k = 1}^p \omega^k m(z \omega^k)$$

$$m(z) = \sum_{n = 0} a_{p^n} m_{\ne p}(z^{p^n})$$
$$ = \sum_{n = 0} a_{p^n} \frac{1}{p} \sum_{k = 1}^p \omega^k m(z^{p^n} \omega^k)$$
$$ = \frac{1}{p} \sum_{k = 1}^p \omega^k \sum_{n = 0} a_{p^n} m(z^{p^n} \omega^k)$$


\subsection{Rambling34}
Let $p \mid N$. Then, and only then:

$$m(z) = \frac{1}{p} \sum_{k = 1}^p \omega^k \sum_{n = 0} m(z^{p^n} \omega^k)$$
$$ = \sum_{n = 0} \frac{1}{p} \sum_{k = 1}^p \omega^k m(z^{p^n} \omega^k)$$


\subsection{Rambling35}
Let: 
$$f(s, t) = \sum_{n = 1}^\infty a_n n^{-s} t^{n - 1}$$
We have: 
$$f(0, t) = m(t)$$
$$f(s, 1) = L(s)$$
$$f(1, t) = M(t)$$


\subsection{Rambling36}
We have:
$$M(z) = \sum_{n = 1} a_n \frac{1}{n} z^{n - 1} = \frac{1}{2\pi} \int_0^{2\pi} m(\sqrt{z}e^{it}) (\sum_{n = 1} 1/n u^{n - 1})(\sqrt{z}e^{-it}) dt$$
We extract:
$$f(u) = \sum_{n = 1} 1/n u^{n - 1} = -\frac{\ln(1 - u)}{u}$$
We then get:
$$M(z) = \sum_{n = 1} a_n \frac{1}{n} z^{n - 1} = \frac{1}{2\pi} \int_0^{2\pi} -\frac{m(\sqrt{z}e^{it})\ln(1 - \sqrt{z}e^{-it})}{\sqrt{z}e^{-it}} dt$$

We also have:
$$m(z) = \sum_{n = 1} a_n \frac{n}{n} z^{n - 1} = \frac{1}{2\pi} \int_0^{2\pi} M(\sqrt{z}e^{it}) (\sum_{n = 1} n u^{n - 1})(\sqrt{z}e^{-it}) dt$$
We extract:
$f(u) = \sum_{n = 1} n u^{n - 1} = \frac{1}{(1 - u)^2}$
We then get
$$m(z) = \frac{1}{2\pi} \int_0^{2\pi} \frac{M(\sqrt{z}e^{it})}{(1 - \sqrt{z}e^{-it})^2} dt$$


\subsection{Rambling37}
Take 11.a1 and do pade approximation (500, 500). At 1/x you get something very similar to m(x).


\subsection{Rambling38}
Take:
$$m_p(z) = 1 + a_p z^p + a_{2p} z^{2p} + \ldots$$
Let $a_n$ be completely multiplicative. We get:
$$m_p(z) = 1 + a_p (a_1 (z^p)^1 + a_{2} (z^{p})^2 + \ldots) = 1 + a_p z^p m(z^p)$$
We see: 
$$\sum_p (m_p(z) - 1) = m(z, a_n \boxtimes \omega)$$
We combine:
$$\sum_p a_p z^p m(z^p) = m(z, a_n \boxtimes \omega)$$
We also have:
$$m(z, a_n \boxtimes \omega) = \frac{1}{2\pi} \int_0^{2\pi} m(\sqrt{z} e^{it}) m(\sqrt{z} e^{-it}; \omega)dt$$
Combining:
$$\sum_p a_p z^p m(z^p) = \frac{1}{2\pi} \int_0^{2\pi} m(\sqrt{z} e^{it}) m(\sqrt{z} e^{-it}; \omega)dt$$


\subsection{Rambling39}

Do pari bestapprPade for 1000 terms for 11.a1. We get a very strange result indeed! We get f(1/x) = Cf'(x)
Can we solve this differential equation?

For 2000 we get f(1/x) = x f(x)

\section{Zeroes}

\subsection{General zeroes}

We have $m(E(N))$ and $m(-E(4N))$ both $0$ for rank 1 and 3
We have $M(E(N))$ and $M(-E(4N))$ both $0$ for rank 2 and 4

\subsection{289.a1}

We have:
$$-0.369432727353547 + 0.596322024707197*I$$
$$0.616384201765624 + 0.470940855106117*I$$
Both with conjugates.

\section{Ideas}

\begin{enumerate}
  \item Jumps in argument of m-function near abs = 1 may correspond to poles!
  \item P-adic m-functions! What happens when you use p-adic numbers or other number in the computations! Does it converge p-adically? Make modifications to make it work?
  \item Integrate in strange paths. For instance bow from 1/2 + 100I to 1/2 - 100I, to compute m.
  \item Integrate mellin inverse at c negative, to capture poles
  \item Use Padé approximation
  \item Use Cesaro summation
  \item Other methods for summing divergent series, on wikipedia
  \item Use p-adic numbers to give value to mfunction
  \item Read up on analytic continuation, wikipedia
  \item Eval m(z)/m(1/z) on circle .99, and use Cauchy integral formula to find this value for other z!
  \item THE ABOVE BUT SYMBOLLICALYY!!! RESIDUE AND ZEROES OF m(1/z)
  \item Use Cauchy integral formula for expanding taylor series at other points, for instance -1/2
  \item Entire wikipedia page on GF transformations
  \item Explore completely multiplicative more, specifically $m_p(z)$ factor out $a_p$
  \item Understand combined L-m-M function using hadamard product and so on.
  \item Find FE for $a_nt^nn^{-s}$ and use to find value at $s = 0$.
  \item Understand the transform zlog(z) on me, we have zero preservance by this transform.
  \item Sum of Mellin inverse should give sum log x + log y = log xy
\end{enumerate}


\end{document}
